%
% LaTeX source of my resume
% =========================
%
% Heavily commented to to fit even LaTeX beginners (hopefully).
%
% See the `README.md` file for more info.
%
% This file is licensed under the CC-NC-ND Creative Commons license.
%
%
% ***************  John... Compile this using 'pdflatex jbmunro4-resume' *************************
%
%



% Start a document with the here given default font size and paper size.
\documentclass[10pt,a4paper]{article}

% Set the page margins.
\usepackage[a4paper,margin=0.75in]{geometry}

% Setup the language.
\usepackage[english]{babel}
\hyphenation{Some-long-word}

% Makes resume-specific commands available.
\usepackage{resume}




\begin{document}  % begin the content of the document
\sloppy  % this to relax whitespacing in favour of straight margins


% title on top of the document
\maintitle{John Munro}{}{Last update on \today}

\nobreakvspace{0.3em}  % add some page break averse vertical spacing

% \noindent prevents paragraph's first lines from indenting
% \mbox is used to obfuscate the email address
% \sbull is a spaced bullet
% \href well..
% \\ breaks the line into a new paragraph
\noindent\href{mailto:jbmunro4.at.gmail.dot.com}{jbmunro4\mbox{}@\mbox{}gmail.com}\sbull
\textsmaller(813) 546-0723\sbull
\href{http://www.linkedin.com/in/jbmunro4}{www.linkedin.com/in/jbmunro4}
\\
755 North Ave NE \#2421\sbull
Atlanta, GA\sbull
30306

\spacedhrule{0.9em}{-0.4em}  % a horizontal line with some vertical spacing before and after

\roottitle{Summary}  % a root section title

\vspace{-1.3em}  % some vertical spacing
\begin{multicols}{2}  % open a multicolumn environment
\noindent \emph{Data nerd with strong development skills, a proven history of designing and building creative data products, an intuition for data architecture, and strong leadership capabilities.}
\\
\\
I enjoy an intensely challenging and creative career in various forms of Information Science that deepen my existing knowledge of Machine Learning, Data Analysis, Artificial Intelligence, Systems Design, and Cognitive Science while broadening my worldview. I want my views to be challenged on a regular basis and to always be learning new things.

I have a passion for understanding the nature of In- telligence and strive to formalize or leverage aspects of Human Intelligence in everything that I build, whether from an AI and Automation perspective or from the perspective of Product Design and Usability.
\end{multicols}


\spacedhrule{0em}{-0.4em}

\roottitle{Experience}

\headedsection  % sets the header for the section and includes any subsections
  {\href{http://www.eatsa.com}{eatsa}}
  {\textsc{San Francisco, CA}} {%
  \headedsubsection
    {Senior Data Engineer \& Lead Data Scientist}
    {Mar \apo16 -- May 17}
    {\bodytext{
      Data Science Lead on Menu Personalization and Dynamic Pricing. Collaborated closely with Product, Design, Engineering, the Food and Operations teams, the Consumer Science team, Analytics, and Executive Leadership to design and implement the initial technical solutions for Menu Recommendations and Frequency-Based Discounts. Designed various pricing models in collaboration with Finance. Designed various Menu Recommendation Algorithms in collaboration with the Science Team, Food Team, and Engineering. Designed and implemented Cibo, a production micro-service built on top of Django to compute and serve personalized menus and prices to our customers.

      \vspace{0.5em}
      Technical Lead on Project Blackbird, a very high priority initiative led by the CEO, to measure and improve operational efficiency. Rapidly designed and implemented a custom Inventory Management Solution after arduously working through complex operational requirements, accounting requirements, executive reporting requirements, and intensely dirty data. Generated product requirements and evaluated inventory management solutions, workforce management solutions, and Enterprise Resource Planning solutions (ERP\apo s). Solely responsible for the vast majority of technical implementation for a highly customized ERP integration. This included eatsa\apo s first production micro-service which was built in Python on top of the \href{https://serverless.com/}{serverless framework}.

      \vspace{0.5em}
      Significantly improved accuracy and removed volatility in the customer-facing Order Delivery ETA provided to customers once they've purchased food through the mobile apps.

      \vspace{0.5em}
      Collaborated tightly with the analytics team to rapidly stand up the initial analytics infrastructure in preparation for Business Intelligence and advanced Machine Learning applications. Provided a centralized data warehouse to correlate and analyze information across a mix of proprietary and third party data sources across the engineering, marketing, product, and science teams. Worked closely with executive leadership and the analytics team to provide frequent board updates and ad-hoc analysis.

      \vspace{0.5em}
      \textit{Environment:} First Data Engineer and Data Scientist at this diverse, early stage fast food startup focused on hardware and software automation toward the goal of healthy, affordable, and quick food. Technolgies included Ruby on Rails, Python, iOS, Redshift, DynamoDB, Redis, PostgreSQL, Looker, Tableau, Django, SciPy, NumPy, Scikit-Learn, and Matplotlib. Machine Learning and Statistical Methods included Bayesian Inference, Collaborative Filtering, Exploratory Data Analysis, Queuing Theory, Time Series Analysis \& Forecasting Linear Regression, and Graphical Models.
    }}
}

\vspace{0.5em}

\headedsection  % sets the header for the section and includes any subsections
  {\href{http://www.anomali.com}{Anomali}}
  {\textsc{Redwood City, CA}} {%
  \headedsubsection
    {Principle Data Scientist, Labs Team}
    {May \apo15 -- Mar 16}
    {\bodytext{

      Individual contributor with a high level of autonomy and a wide range of responsibilities across the entire company. My tasks frequently required Core Product Engineering, Threat Intelligence Research, Exploratory Data Analysis, Data Visualization, Business Intelligence (BI) Metric Design, End to End Product Design, and User Experience (UX) Design.

      \vspace{0.5em}
      Designed and engineered the next iteration of Anomali's Machine Learning-based Domain and IP Reputation engine, the core product's central use case. This new, highly distributable micro-service included very significant improvements in model accuracy, response time, and throughput. It was designed to auto-tune the underlying Machine Learning model and report thorough goodness metrics for regular, semi-automated updates as feature distributions, data sources, and code change.

      \vspace{0.5em}
      Worked on targeted research projects with security analysts, vetted new data sources, designed gamification mechanisms for our online communities, and contributed to the sales team by integrating Anomali's many data sources to give a \(360^{\circ}\) view of customers as they move through the sales process and make use of Anomali's SaaS security platform.

      \vspace{0.5em}
      \textit{Environment:} Intermediary between Labs (Threat Research Team) and the core product engineering teams. Technologies included Python, SciPy, NumPy, Scikit-Learn, Matplotlib, Django, PostgreSQL, Splunk, the ELK Stack (ElasticSearch, LogStash, Kibana), Looker, HTML, CSS, JavaScript, jQuery, Angular, R, and GGPlot2. Machine Learning and Statistical Methods included Random Forests (Decision Forests), k-NN's, DBScan, SVM's, Time Series Analysis, Time Series Forecasting, Feature Selection, Stratified Sampling, and Simple Random Sampling.
    }}
}

\vspace{0.5em}

\headedsection  % sets the header for a subsection and contains usually body text
  {\href{http://www.endgame.com}{Endgame}}
  {\textsc{San Francisco, CA}} {%
  \headedsubsection
    {Lead Data Scientist, Hiring Manager}
    {Jun \apo14 -- May \apo15}
    {\bodytext{
      Built a diverse team of product-focused Data Scientists in an effort to create various analytical models for a variety of Cyber Security applications. Implemented cross-functional collaboration to evangelize our capabilities and learn more about potential use cases across the company. Initiated a culture of healthy peer review along with multiple successful initiatives to improve the team\apo s security domain knowledge.

      \vspace{0.5em}
      Designed and contributed to a Data Science Platform that supports analytic R\&D in both Batch and Streaming environments at scale. This platform services a new product aimed at giving System and Security Administrators visibility into cloud based production environments to identify anomalous behavior indicative of security threats, misconfigurations, or inappropriate resource allocation. Our focus was on Multidimensional Anomaly Detection using a combination of Supervised and Unsupervised Machine Learning and Graph-Theoretic approaches to perform Offline (Batch) Clustering and Online (Real Time Streaming) Classification or Regression.

      \vspace{0.5em}
      \textit{Environment:} Small, distributed team working toward Endgame's first commercial product. Technologies included Spark, Scala, PySpark, Storm, Kafka, Redis, Java, Python, NumPy, SciPy, Scikit-Learn, Matplotlib, Pandas, R, GGPlot2, d3.js, HTML, JavaScript, CSS, jQuery, Angular, React, PostgreSQL, ElasticSearch, Hadoop, and Cassandra.  Machine Learning and Statistical techniques were largely adapted for our specific use cases, inspired by traditional algorithms such as Random Forests (Decision Forests), SVM, k-NN, k-Medoids, and Time Series Forecasting.
    }}
  }

\headedsection  % sets the header for a subsection and contains usually body text
  {}
  {\textsc{Washington, DC}} {%
  \headedsubsection
    {Senior Backend Developer, Tech Lead}
    {Oct \apo12 -- Jun \apo14}
    {\bodytext{
      Researched, designed, and implemented a full stack product to identify trends in large, High Dimensional Data sets using Unsupervised Machine Learning methods to aid in data exploration. Iterated closely with the customer to define their needs and determine appropriate solutions which provide a significant advantage over their old methods and redefine how they think about their specific problems. As Tech Lead, I hired, onboarded, and led a small team of developers to support the product going forward.

      \vspace{0.5em}
      \textit{Environment:} Small, customer facing development team within an early stage startup focused on building crucial products for a very high profile governement use case. Technologies included Python, Django, R, GGPlot2, NumPy, SciPy, Scikit-Learn, Matplotlib, Pandas, d3.js, HTML, JavaScript, CSS, jQuery, Angular, PostgreSQL, ElasticSearch, Redis, Hadoop, and Pig. Machine Learning and Statistical Techniques included DBScan, k-NN, k-Medoids, Random Forests (Decision Forests), Cluster Analysis, Simple Random Sampling, Stratified Sampling, Principle Component Analysis (PCA), Feature Selection, Random Forrests, etc.
    }}
  }

\headedsection  % sets the header for a subsection and contains usually body text
  {}
  {\textsc{Atlanta, GA}} {%
  \headedsubsection
    {Malware Analysis Engineer (Data Scientist)}
    {Oct \apo11 -- Oct \apo12}
    {\bodytext{
      Designed, built, patented, and \href{http://resources.sei.cmu.edu/asset_files/Presentation/2013_017_101_51242.pdf}{presented Clairvoyant Squirrel}, a novel real time classification system which utilizes Random Forests and other Machine Learning techniques over Big Data-scale network sensor output to score and classify the malicousness of domain names.

      \vspace{0.5em}
      Researched and implemented a wide variety of AI, Machine Learning and Statistical Models to classify and categorize malicious characteristics of malware executables, TCP packet headers, malware domain names at various points in the DNS hierarchy, and Botnet Command and Control (C\&C or C2) communication patterns.

      \vspace{0.5em}
      Frequently performed Exploratory Data Analysis, Data Visualization, Statistical Sampling, and ETL using a wide variety of tools, languages, and frameworks. Techniques included Supervised and Unsupervised Machine Learning, Classification, Regression, Cluster Analysis, Natural Language Processing (NLP), Time Series Analysis, Time Series Forecasting, and Dimensionality Reduction.

      \vspace{0.5em}
      \textit{Environment:} Sole data scientist on a small research team focused on developing cutting edge security products within this early stage startup. Frequently used R, GGPlot2, Weka, Java, Python, NumPy, SciPy, Pandas, Matplotlib, Scikit-Learn, NLTK, Hadoop, Pig, Redis, ElasticSearch, MongoDB, RabitMQ, and PostgreSQL.  Statistical and Machine Learning techniques included Random Forests (Decision Forests), k-NN's, SVM's, Neural Networks, Naive Bayes, Decision Trees, k-Medoids, Linear and Logistic Regression, Spectral Clustering, DBScan, Principle Component Analysis (PCA), Latent Dirichlet Allocation (LDA), TF-IDF, Simple Random Sampling, and Stratified Sampling.
    }}
  }

\vspace{0.5em}

\headedsection
  {\href{https://www.linkedin.com/company-beta/4014/}{Autonomy}{ (since purchased by HP)}}
  {\textsc{Atlanta, GA}} {%
  \headedsubsection
    {Technology Specialist}
    {Jan \apo11 -- Oct \apo11}
    {\bodytext{
      Rapidly implemented Full Stack Proof of Concepts for targetted customer demos of Autonomy\apo s Machine Learning-based enterprise search technology and vertically integrated solutions. Acted as intermediary between customers and engineering to communicate business requirements and debug solutions. Worked internationally on a high profile customer site to configure, deploy, and test new Autonomy products in a large production environment.

      \vspace{0.5em}
      \textit{Environment:} Very fast paced and high pressure sales environment utilizing proprietary backend and frontend components along with Perl, Python, HTML, JavaScript, and CSS.  80\% travel and 20\% remote.
    }}
  }


\vspace{-0.2em}
\begin{center}
  \emph{\small Please refer to my \href{http://www.linkedin.com/in/jbmunro4}{Linked-in profile} for a more complete list of work experiences along with recommendations.}
\end{center}


\spacedhrule{-0.2em}{-0.4em}

\roottitle{Education}

\headedsection
  {\href{http://www.gatech.edu}{Georgia Institute of Technology}}
  {\textsc{Atlanta, GA}} {%
  \headedsubsection
    {Bachelor degree in Computer Science \& Minor in Psychology}
    {2006 -- 2010}
    {\bodytext{Specialized heavily in Artificial Intelligence, Cognitive Neuroscience and Computer Networking. Completed 3 years of Undergraduate Research focused on AI, Automated Story Telling, and Game Design with a thesis on computational models for automating machinima generation. Took courses in Machine Learning, Statistics, Experimental Design, Robotics, Computer Networking, Cognitive Psychology, and Neuropsychology.}}
}


\spacedhrule{0.5em}{-0.4em}

\headedsection  % sets the header for a subsection and contains usually body text
  {Skills Overview}
  {} {%
  \headedsubsection
    {Data Science and Engineering Skills:}
    {\bodytext{
    % Random Forests (Decision Forests), k-NN, SVM, Decision Trees, K-Means, K-Medoids, DBScan, Hierarchical Clustering, Self Organizing Maps, Neural Networks, Principle Components Analysis (PCA),
    % Information Gain Analysis, Linear Regression, Logistic Regression, Holtz-Winters Time Series Forecasting, etc.}

      \textit{Machine Learning:} Random Forests (Decision Forests), k-NN, SVM, Decision Trees, K-Means, K-Medoids, DBScan, Hierarchical Clustering, Self Organizing Maps, Neural Networks, Principle Components Analysis (PCA),

      \vspace{0.5em}
      \textit{Statistics:} Information Gain Analysis, Linear Regression, Logistic Regression, Holtz-Winters Time Series Forecasting, etc.

      \vspace{0.5em}
      \textit{And more generally:} Supervised Machine Learning (Classification, Regression), Unsupervised Machine Learning (Clustering), Data Visualization, Statistical Hypothesis Testing, Statistical Distributions, Graph Analytics, Time Series Forecasting, Natural Language Processing (NLP),


      \textit{Software Engineering Skills:} Test Driven Development (TDD), Production Software Development, Rapid Prototyping

      \textit{Leadership Experience:} Product Tech Lead, Big Data Software Architecture, Team Management and Hiring, Cross Functional Collaboration, Project Management, Project Planning
    }}
}

\vspace{0.5em}
\inlineheadsection
  {Languages:}
  % {Python, R, Bash, Javascript, HTML, CSS, SQL, Java, Scala, C, C++, Matlab, Pig, etc.}
  {Bash, C, CSS, C++, HTML, Java, JavaScript, Matlab, Perl, Pig, Python, R, Scala, CSS, SQL}

\vspace{0.5em}
\inlineheadsection
  {Frameworks, tools, and libraries:}
  {
    \textit{Python:} Django, NumPy, SciPy, Scikit-Learn, Matplotlib, PySpark, GGPlot2,
    \textit{JavaScript:} d3.js, jQuery, Angular, React
    \textit{Java:} Weka, Storm, Spark
    \textit{Scala:} Spark
  }

\vspace{0.5em}
\inlineheadsection
  {Databases and Big Data Cloud Technologies:}
  {
    \textit{SQL and No-SQL Databases:} PostgreSQL, ElasticSearch, MongoDB, Redis
    \textit{Queuing Technologies:} RabitMQ, Kafka
    \textit{Cloud Technologies:} Amazon AWS, s3, ec2, Redshift, DynamoDB, Hadoop, HDFS, Map Reduce, Spark, Storm
  }

\vspace{0.5em}
\inlineheadsection
  {Business Analyst Tools:}
  {Looker, Tableau}

\vspace{0.5em}
\inlineheadsection
  {Operating Systems:}
  {Centos, iOS, Linux, MacOS, Ubuntu, Unix}


% k-NN, Naive Bayes, SVM, Decision Forests, etc.
% Experience with common data science toolkits, such as R, pyspark, Weka, NumPy, MatLab, etc . Excellence in at least one of these is highly desirable

% Experience with data visualisation tools, such as D3.js, GGplot, etc.
% Proficiency in using query languages such as SQL, Hive, Spark
% Experience with NoSQL databases, such as MongoDB, Cassandra, HBase
% Good applied statistics skills, such as distributions, statistical testing, regression, etc.


\spacedhrule{1.6em}{-0.4em}

\roottitle{Interests}

\inlineheadsection
  {Non-exhaustive and in alphabetical order:}
  {cooking, data visualization, eating, education, game design and development, robotics, snowboarding, UX and product design}



% ----------------------------------------------------------------------------

\roottitle{Skills Overview}

\inlineheadsection  % special section that has an inline header with a 'hanging' paragraph
  % {Applied and theoretical algorithmic expertise:}
  {Data Science and Engineering Skills:}
  {
  % Random Forests (Decision Forests), k-NN, SVM, Decision Trees, K-Means, K-Medoids, DBScan, Hierarchical Clustering, Self Organizing Maps, Neural Networks, Principle Components Analysis (PCA),
  % Information Gain Analysis, Linear Regression, Logistic Regression, Holtz-Winters Time Series Forecasting, etc.}

    \textit{Machine Learning:} Random Forests (Decision Forests), k-NN, SVM, Decision Trees, K-Means, K-Medoids, DBScan, Hierarchical Clustering, Self Organizing Maps, Neural Networks, Principle Components Analysis (PCA),

    \vspace{0.5em}
    \textit{Statistics:} Information Gain Analysis, Linear Regression, Logistic Regression, Holtz-Winters Time Series Forecasting, etc.

    \vspace{0.5em}
    \textit{And more generally:} Supervised Machine Learning (Classification, Regression), Unsupervised Machine Learning (Clustering), Data Visualization, Statistical Hypothesis Testing, Statistical Distributions, Graph Analytics, Time Series Forecasting, Natural Language Processing (NLP),


    \textit{Software Engineering Skills:} Test Driven Development (TDD), Production Software Development, Rapid Prototyping

    \textit{Leadership Experience:} Product Tech Lead, Big Data Software Architecture, Team Management and Hiring, Cross Functional Collaboration, Project Management, Project Planning
  }

\vspace{0.5em}
\inlineheadsection
  {Languages:}
  % {Python, R, Bash, Javascript, HTML, CSS, SQL, Java, Scala, C, C++, Matlab, Pig, etc.}
  {Bash, C, CSS, C++, HTML, Java, JavaScript, Matlab, Perl, Pig, Python, R, Scala, CSS, SQL}

\vspace{0.5em}
\inlineheadsection
  {Frameworks, tools, and libraries:}
  {
    \textit{Python:} Django, NumPy, SciPy, Scikit-Learn, Matplotlib, PySpark, GGPlot2,
    \textit{JavaScript:} d3.js, jQuery, Angular, React
    \textit{Java:} Weka, Storm, Spark
    \textit{Scala:} Spark
  }

\vspace{0.5em}
\inlineheadsection
  {Databases and Big Data Cloud Technologies:}
  {
    \textit{SQL and No-SQL Databases:} PostgreSQL, ElasticSearch, MongoDB, Redis
    \textit{Queuing Technologies:} RabitMQ, Kafka
    \textit{Cloud Technologies:} Amazon AWS, s3, ec2, Redshift, DynamoDB, Hadoop, HDFS, Map Reduce, Spark, Storm
  }

\vspace{0.5em}
\inlineheadsection
  {Business Analyst Tools:}
  {Looker, Tableau}

\vspace{0.5em}
\inlineheadsection
  {Operating Systems:}
  {Centos, iOS, Linux, MacOS, Ubuntu, Unix}


% k-NN, Naive Bayes, SVM, Decision Forests, etc.
% Experience with common data science toolkits, such as R, pyspark, Weka, NumPy, MatLab, etc . Excellence in at least one of these is highly desirable

% Experience with data visualisation tools, such as D3.js, GGplot, etc.
% Proficiency in using query languages such as SQL, Hive, Spark
% Experience with NoSQL databases, such as MongoDB, Cassandra, HBase
% Good applied statistics skills, such as distributions, statistical testing, regression, etc.


\spacedhrule{1.6em}{-0.4em}

\roottitle{Interests}

\inlineheadsection
  {Non-exhaustive and in alphabetical order:}
  {cooking, data visualization, eating, education, game design and development, robotics, snowboarding, UX and product design}


\end{document}
